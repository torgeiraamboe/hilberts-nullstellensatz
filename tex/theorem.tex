

\section{Hilbert's nullstellensatz}

This is by no means the shortest proof of this theorem, 
nor is it the most elegant. 
I chose to do it this way because it's more involved and has a couple extra moving parts. 
It was also a good excuse to look into Jacobson rings. 

\begin{theorem}[Hilbert's nullstellensatz]
Let $k$ be an algebraically closed field, 
$A$ be the polynomial ring $A = k[t_1, \ldots, t_n]$ and $\mathfrak{a}$ be an ideal of $A$. 
Then
\begin{equation*}
    I(Z(\mathfrak{a})) = \sqrt{\mathfrak{a}}
\end{equation*}
\end{theorem}

We are going to do the proof in three smaller steps. 
\begin{enumerate}
    \item Proving that ideals of the form $\mathfrak{m}_a = (t_1 - a_1, \ldots t_n - a_n)$ where $a\in k^n$ are the only maximal ideals in $A$. 
    This is known as the weak Hilbert's nullstellensatz. 
    \item Proving that the radical of an ideal $\mathfrak{b}$ in a finitely generated $k$-algebra $B$ is equal to the intersection of the maximal ideals in $B$ that contain $\mathfrak{b}$. 
    This shows that all finitely generated algebras over a field is a Jacobson ring. 
    \item Deduce the result. 
\end{enumerate}

\subsection{Part 1}
\begin{lemma}
Let $\mathfrak{m}_a = (t_1 - a_1, \ldots t_n - a_n)$ where $a\in k^n$. 
The ideals of this form are the only maximal ideals in $A$.
\end{lemma}
\begin{proof}
Let $a \in k^n$. 
We define a the evaluation morphism as follows: 
\begin{align*}
    e_{a}: k[t_1,\ldots, t_n]&\longrightarrow k \\
    f &\longmapsto f(a) .
\end{align*}
Note that it is a surjective $k$-algebra homomorphism and since $k$ is algebraically closed, 
it has kernel $\mathfrak{m}_{a}$. 
Let $\mathfrak{m}$ be a maximal ideal in $k[t_1, \ldots, t_n]$. Then $k[t_1, \ldots, t_n]/\mathfrak{m}$ is a finitely generated field extension of $k$. 
By Zariski's lemma, 
$k[t_1, \ldots, t_n]/\mathfrak{m}$ is in fact a finite field extension, 
better known as a finite dimensional vector space. 
Since $k$ is algebraically closed, 
there is an isomorphism of $k$-algebras
\begin{equation*}
    k[t_1, \ldots, t_n]/\mathfrak{m}\longrightarrow k .
\end{equation*}
Now, let $a_i$ denote the image of $t_i$. 
Then we get that $\mathfrak{m}_{a}\subseteq \mathfrak{m}$, 
which implies $\mathfrak{m}_{a} = \mathfrak{m}$ since $\mathfrak{m}_{a}$ is a maximal ideal. 
\end{proof}

\subsection{Part 2}
\begin{lemma}
Let $k$ be an algebraically closed field, 
$B$ be a finitely generated $k$-algebra and $\mathfrak{b}$ be an ideal in $B$. 
Then we have
\begin{equation*}
    \sqrt{\mathfrak{b}} = \bigcap_{\mathfrak{b}\subseteq\mathfrak{m}}\mathfrak{m}.
\end{equation*}
where $\mathfrak{m}$ are the maximal ideals in $B$.
\end{lemma}
\begin{proof}
First, 
we note that the projection $\pi:B\rightarrow B/\mathfrak{b}$ induces bijections between the sets
\begin{itemize}
    \item prime ideals in $B/\mathfrak{b}$ and prime ideals in $B$ that contain $\mathfrak{b}$,
    \item maximal ideals in $B/\mathfrak{b}$ and maximal ideals in $B$ that contain $\mathfrak{b}$,
    \item radical ideals in $B/\mathfrak{b}$ and radical ideals in $B$ that contain $\mathfrak{b}$.
\end{itemize}
Hence we only need to prove the statement for $\mathfrak{b}=(0)$, 
and since it is clear that $\sqrt{(0)}$ is contained in every maximal ideal because $\sqrt{(0)}$ consists of all nilpotent elements, 
we only need to show that every element not contained in $\sqrt{(0)}$ is not contained in some maximal ideal. 

Let $f\in B$ be non-nilpotent, 
i.e. $f\in \sqrt{(0)}$. 
This implies that $$B_f\cong B[t]/(ft-1)$$ is a non-trivial $k$-algebra, 
hence it has a maximal ideal $\mathfrak{m}$. 
Consider the morphism $\phi: B \longrightarrow B_f$. 
This is a morphism of finitely generated $k$-algebras, 
and by Zariski's lemma, 
$k\subseteq B/\phi^{-1}(\mathfrak{m})\subseteq B_f/\mathfrak{m}$ is a finite extension, 
and hence $k\subseteq B/\phi^{-1}(\mathfrak{m})$ is an integral extension. 
Since $k$ is a field, it is a field itself. 
This gives us that the inverse image of a maximal ideal is again a maximal ideal, 
i.e. $\phi^{-1}(\mathfrak{m})$ is a maximal ideal of $B$. 
But this ideal can't contain $f$. 
Hence we have shown that every non-nilpotent element is not contained in all maximal ideals. 
\end{proof}


\subsection{Part 3}
We now deduce the result. 
\begin{proof}
Let $a \in k^n$. 
First, 
note that $a \in Z(\mathfrak{a})$ if and only if $\mathfrak{a}\subseteq \mathfrak{m}_{a}$. 
Hence, 
the maximal ideals containing $\mathfrak{a}$ is just the maximal ideals $\mathfrak{m}_{a}$ such that $a \in Z(\mathfrak{a})$. 
In the second step we showed that the radical of an ideal was equal to the intersection of all maximal ideals containing it, 
hence we have 
\begin{equation*}
    \sqrt{\mathfrak{a}} = \bigcap_{a \in Z(\mathfrak{a})}\mathfrak{m}_{a}. 
\end{equation*}
For the final part, 
we have for $f\in k[t_1,\ldots,t_n]$ and $a\in k^n$ that $f(x) = 0$ if and only if $f \in \mathfrak{m}_{a}$. 
Hence we have for subsets $V\subseteq k^n$ that $I(V) = \bigcap_{a \in V}\mathfrak{m}_{a}$. 
And since $a \in Z(\mathfrak{a})$ if and only if $\mathfrak{a}\subseteq \mathfrak{m}_{a}$ we have finally

\begin{align*}
    I(Z(\mathfrak{a})) 
    &= \bigcap_{a \in Z(\mathfrak{a})}\mathfrak{m}_{a} \\
    &= \sqrt{\mathfrak{a}}.
\end{align*}
And by that, we are done!
\end{proof}




