

\section{Preliminaries}
All rings mentioned are commutative and have a multiplicative identity, 
referred to as $1$. 

We will try not to assume that the reader knows everything, 
but we will assume general knowledge of algebra and some introductory commutative algebra like definitions of rings, 
modules, algebras etc. 

\begin{theorem}[Noether's normalization lemma]
Let $k$ be a field and $B$ a non-trivial finitely generated $k$-algebra. 
Then there exists a non-negative integer $d$ and algebraically independent elements $y_1, ..., y_d$, 
i.e. elements with no relations, 
in $B$ such that $B$ integral over the polynomial ring $k[y_1, ..., y_d]$.
\end{theorem}
\begin{proof}
Note that the following proof is heavily based on the proof from Mumford's ``The red book''. 

Since $B$ is finitely generated the proof is done by induction on the number of generators, $m$. 

$m=0$: 
In this case $A=k$ and we are done. 

Assume $m>0$ and that the lemma holds for all $k$-algebras generated by $m-1$ or fewer elements.

Now, let $B$ be generated by $y_1, \ldots, y_{m}$. 
If there is no relations on the generators, 
i.e. they are algebraically independent, then
\begin{equation*}
    B \cong k[y_1, \ldots, y_{m}]
\end{equation*}
and we are done. 
Hence we assume that $f$ is a relation on the $m$ generators, 
i.e. a polynomial equation
\begin{equation*}
    f(y_1, \ldots y_m) = 0. 
\end{equation*}
Let $r$ be an integer. 
We are going to determine this later. 
Set $z_1 = y_1$ and $z_i = y_i - y_1^{r^{i-1}}$ for $2\leq i \leq m$.
Note that the $z_i$'s also are generators of $B$.
Then we can rewrite the relation as
\begin{equation*}
    f(z_1, z_2+y_1^r, z_3+y_1^{r^2}, \ldots z_m+y_1^{r^{m-1}}) = 0.
\end{equation*}
For $a\in k$, 
the highest term of $y_1$ appearing in $a\prod_1^m (z_i + z_1^{r^{i-1}})^{\alpha_i}$ looks like $ay_1^{\alpha_1+r\alpha_2+\ldots+r^{m-1}\alpha_m}$. 
Hence, if $r$ is larger than all of the $\alpha_i$'s appearing in $f$, 
then the highest term of $y_1$ appearing in $f(z_1, z_2+y_1^r, z_3+y_1^{r^2}, \ldots z_m+y_1^{r^{m-1}})$ also has the same form as the one above, 
which means that $y_1$ is integral over $k[y_2, \ldots, y_m]$. 

Since $y_{i}=z_{i}+y_{1}^{r^{i-1}}$ are also integral over the same ring, 
the $z_i$'s generate $B$, and being an integral extension is a transitive property, 
we get that B is integral over $k[y_2, \ldots, y_m]$. 
Since $k[y_2, \ldots, y_m]$ is generated by $m-1$ elements, 
we can apply the inductive hypothesis, and hence we are done.
\end{proof}

%Note that Noether's normalization lemma also can be formulated, 
%and hence is equivalent to the statement that there exists a finite injective $k$-algebra homomorphism $k[y_1, \ldots y_d]\longrightarrow B$ for some $d\in \mathbb{N}$.

We want to apply Noether's normalization lemma to prove Zariski's lemma. 
In the proof we present we use a result stating that integral ring extensions preserve Krull dimension. 
We actually only use the fact that if a ring with Krull dimension zero is an integral ring extension of another ring, 
then this ring also has Krull dimension zero. 
The more general statement also is true, 
but we only prove a more strict statement that we need. 

We remind ourself what we mean by Krull dimension.
\begin{definition}[Krull dimension]
Let $A$ be a ring. 
The Krull dimension of $A$ is defined as the supremum over the length of all chains of prime ideals in $A$.
\end{definition}

\begin{proposition}
Let $S$ be a ring and $T$ be an integral ring extension of $S$. 
Then the Krull dimension of $S$ is lower or equal to the Krull dimension of $T$. 
\end{proposition}
\begin{proof}
Let $\mathfrak{p}_0\subsetneq \cdots \subsetneq \mathfrak{p}_n$ be a chain of prime ideals in $S$. 
By the Going-up theorem we can extend any chain of length less than $n$, 
$\mathfrak{q}_0\subsetneq \cdots \subsetneq \mathfrak{q}_m$ such that $\mathfrak{q}_i\cap S = \mathfrak{p}_i$, 
for $i=1,\ldots, m$, 
to a chain of prime ideals $\mathfrak{q}_0\subsetneq \cdots \subsetneq \mathfrak{q}_n$ in $T$, 
such that $\mathfrak{q}_i\cap S = \mathfrak{p}_i$ for $i=1,\ldots, n$. 
Hence the Krull dimension of an integral ring extension can't get any lower than the Krull dimension of the ring it extends. 
\end{proof}

Note that we here used the so called Going-up theorem. 
This theorem proves exactly what is stated in the proof of the proposition, 
i.e. that we can extend chains of prime ideals in integral extensions. 
We will not prove this theorem, 
but we refer the reader to the proof in Introduction to commutative algebra by Atiyah and MacDonald. 

This immediately implies what we need, 
i.e. that if a ring with Krull dimension zero is an integral ring extension of another ring, 
then this ring also has Krull dimension zero. 

\begin{corollary}[Zariski's lemma]
Let $k$ be a field, and $K$ be a field extension that is finitely generated as an algebra over $k$. 
Then $K$ is a finite field extension, 
i.e. a finite dimensional $k$-vectorspace. 
\end{corollary}
\begin{proof}
By Noether's normalization lemma, 
$K$ is integral over the ring $k[y_1, \ldots y_d]$, 
where $y_1, \ldots y_d$ are the elements of $K$ that are algebraically independent over $k$. 
Since $K$ is a field, it has Krull dimension $0$, 
and since integral ring extensions preserve Krull dimension, 
the polynomial ring $k[y_1, \ldots y_d]$ must have Krull dimension zero, 
i.e. $d=0$. Hence $K$ is a finitely generated module over $k$, 
i.e. a finite dimensional $k$-vectorspace. 
\end{proof}

Now, before we tackle the actual proof of Hilbert's nullstellensatz, 
we need to chose a formulation. 
There are many formulations of the theorem, 
all hopefully equivalent, 
but we have chosen one of the more standard and iconic ones. 
The formulation is based on some language from algebraic geometry, 
hence we need some definitions. 

\begin{definition}[Radical]
Let A be a ring and $\mathfrak{a}$ an ideal in A. 
We define the radical of $\mathfrak{a}$ to be the set
\begin{equation*}
    \sqrt{\mathfrak{a}} = \{ f\in A \ | \ \exists n \in \mathbb{N} \text{ s.t. } f^n \in \mathfrak{a} \}.
\end{equation*}
We say an ideal $\mathfrak{a}$ is a radical ideal if $\mathfrak{a}=\sqrt{\mathfrak{a}}$. 
\end{definition}

\begin{definition}[Zero locus]
Let $k$ be an algebraically closed field, 
$A$ be the polynomial ring $A = k[t_1, \ldots, t_n]$ and let $S \subset A$ be a subset. 
We define the zero locus of $S$ to be the set 
\begin{equation*}
    Z(S) = \{ x\in k^n \ | \ f(x)=0, \ \forall f\in S\}.
\end{equation*}
\end{definition}
If a subset $V\subset k^n$ has the form $V=Z(S)$ for some $S\subset A$, 
then we call $V$ an affine algebraic variety.

\begin{definition}[Vanishing set]
Let $V\subset k^n$. We define the vanishing set of $V$ to be the set
\begin{equation*}
    I(V) = \{ f\in A \ | \ f(x)= 0, \ \forall x\in V\}.
\end{equation*}
\end{definition}